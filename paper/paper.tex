\documentclass{llncs}

\usepackage[utf8]{inputenc}
\usepackage[T1]{fontenc}

\usepackage[hidelinks]{hyperref}

\usepackage[links]{agda}

\usepackage{amsmath}
\usepackage{amsfonts}
\usepackage{amssymb}

\usepackage{newunicodechar}
\newunicodechar{≡}{\ensuremath{\equiv}}
\newunicodechar{ℕ}{\ensuremath{\mathbb{N}}}
\newunicodechar{∘}{\ensuremath{\circ}}
\newunicodechar{∅}{\ensuremath{\varnothing}}
\newunicodechar{⇒}{\ensuremath{\Rightarrow}}
\newunicodechar{∋}{\ensuremath{\ni}}
\newunicodechar{∷}{\ensuremath{\mathop{::}}}
\newunicodechar{⊢}{\ensuremath{\vdash}}
\newunicodechar{∀}{\ensuremath{\forall}}
\newunicodechar{₁}{\ensuremath{{}_1}}
\newunicodechar{₂}{\ensuremath{{}_2}}
\newunicodechar{→}{\ensuremath{\to}}
% \newunicodechar{Γ}{\ensuremath{\texttt{\Gamma}}}
% \newunicodechar{λ}{\ensuremath{\texttt{\lambda}}}

\usepackage{cite}
% \usepackage{DejaVuSansMono}
\usepackage{alphabeta} % Make greek letter macros also work in text-mode.

\usepackage{microtype}
\DisableLigatures[-]{encoding=T1}

\title{Kit Theory}

\begin{document}
  \maketitle

  \begin{abstract}
    Variable binding and substitution are a long-standing nuisance for mechanized
    metatheory of programming languages. A common approach is to represent
    variables as DeBruijn Indices, where substitution is defined in terms of
    variable renaming. If a language has $n$ different kinds of variables, and
    $m$ different kinds of terms, we potentially need to define both
    substitution and renaming for all $m\cdot n$ combinations and prove the
    theory of those operations both in isolation and with respect to each other.
    McBride introduced kits to define both substitution and renaming by a
    single term traversal for intrinsically typed terms.\cite{mcbride2005kits}

    We propose a technique to derive the implementation and metatheory of
    substitution and renaming for all $m\cdot n$ combinations of variables and terms
    from a single term traversal and a few simple lemmas.
    We use a slight generalization of McBride's kits but for extrinsic typing,
    which decouples type preservation from scope preservation and the
    theory of substitution. Hence, the technique is also suitable for
    non-standard typing relations like dependent types and type state.
    We provide a well-documented implementation of the technique as an Agda
    library and show its use in several case studies including System F.
  \end{abstract}

  % The following line will be replaced by make with an \input{} for each
  % .lagda.tex file:

  % SED MARKER

  % \bibliographystyle{ACM-Reference-Format}
  \bibliographystyle{abbrv}
  \bibliography{paper}

  \clearpage
  \appendix
\end{document}