% \documentclass[sigplan, screen]{acmart}
% \documentclass[sigplan, screen, anonymous, review, authordraft]{acmart}
% \documentclass[sigplan,10pt,anonymous,review]{acmart}\settopmatter{printfolios=true,printccs=false,printacmref=false}
\documentclass[sigplan,10pt]{acmart}

\usepackage{agda-unicode}

\bibliographystyle{ACM-Reference-Format}

\title{Kit Theory}

\author{Hannes Saffrich}
\orcid{0000-0002-1825-0097}                 %% \orcid is optional
\affiliation{
  % \department{Department1}                %% \department is recommended
  \institution{University of Freiburg}      %% \institution is required
  \country{Germany}                         %% \country is recommended
}
\email{saffrich@informatik.uni-freiburg.de} %% \email is recommended

% \author{Peter Thiemann}
% \orcid{0000-0002-9000-1239}                 %% \orcid is optional
% \affiliation{
%   % \department{Department2a}               %% \department is recommended
%   \institution{University of Freiburg}      %% \institution is required
%   \country{Germany}                         %% \country is recommended
% }
% \email{thiemann@informatik.uni-freiburg.de} %% \email is recommended

\usepackage{tcolorbox}

\newenvironment{LibCode*}{%
  \begin{tcolorbox}[%
    colframe=white,%
    boxrule=0.0pt,%
    top=2.5pt,%
    left=2.5pt,%
    bottom=2.5pt,%
    right=2.5pt,%
    boxsep=0pt%
  ]\vspace{-0.2\baselineskip}%
}{%
  \vspace{-1\baselineskip}%
  \end{tcolorbox}%
}

\newenvironment{ExampleCode*}{%
  \begin{tcolorbox}[%
    colframe=white,%
    colback=yellow!5,%
    boxrule=0.0pt,%
    top=2.5pt,%
    left=2.5pt,%
    bottom=2.5pt,%
    right=2.5pt,%
    boxsep=0pt%
  ]\vspace{-0.2\baselineskip}%
}{%
  \vspace{-1\baselineskip}%
  \end{tcolorbox}%
}

\newcommand*\LibCode[1]{\begin{LibCode*}{#1}\end{LibCode*}}
\newcommand*\AppCode[1]{{#1}}
\newcommand*\ExampleCode[1]{\begin{ExampleCode*}{#1}\end{ExampleCode*}}

\begin{document}
  \begin{abstract}
    Variable binding and substitution are a long-standing nuisance for mechanized
    metatheory of programming languages. A common approach is to represent
    variables as DeBruijn Indices, where substitution is defined in terms of
    variable renaming. If a language has $n$ different kinds of variables, and
    $m$ different kinds of terms, we potentially need to define both
    renaming and substitution for all $m\cdot n$ combinations and prove the
    theory of those operations both in isolation and with respect to each other.

    % McBride introduced kits to define both substitution and renaming by a
    % single term traversal for intrinsically typed terms.\cite{mcbride2005kits}

    We propose a technique to derive the implementation and metatheory of
    renaming and substitution for all $m\cdot n$ combinations of variables and terms
    from a single term traversal and three simple lemmas for extrinsically typed
    but intrinsically scoped terms.
    We demonstrate this technique by formalizing type soundness of System F with
    very parallel substitutions, which replace both type and term variables at
    once, enabling a single substitution-preserves-typing lemma.
    We provide a well-documented implementation of the technique as an Agda
    library.

    As the technique is based on extrinsic typing, the metatheory of renaming
    and substitution is decoupled from type preservation.
    This has the downside, that type preservation of renaming and substitution
    are not for free, but also the upside, that the library can be directly
    instantiated for object languages with non-standard typing relations like
    dependent types, type state, or resource environments.
  \end{abstract}

  \maketitle

  % The following line will be replaced by make with an \input{} for each
  % .lagda.tex file:

  % SED MARKER

  \AgdaNoSpaceAroundCode{}

  \section{Introduction}
  Formalizing metatheory of programming languages in a proof assistant 
  requires to deal with variables, binders and substitution.

  In most languages those constructs behave uniformly,
  i.e. substitution is only concerned with variables and binders and
  behaves homomorphic on all other syntax constructors.

  Intrinsic typing 


  \begin{enumerate}
  \item DeBruijn
  \item Intrinsic vs Extrinsic
  \item Reduction under binder
  \end{enumerate}

  \subsection{Contributions}
  \begin{enumerate}
  \item
    a multi-sorted approach to extrinsic typing;
  \item
    a multi-sorted extension of McBride's kits to unify renamings and substitutions and their metatheory;
  \item
    a kit abstraction for composition and it's metatheory, unifying the four combinations of renamings and substitutions;
  \item
    a kit abstraction for typing relations, unifying type preservation of renaming and substitution into a single lemma;
  \item
    an Agda-implementation featuring derived term-kits, representation
    independence for substitutions and typing contexts, heterogeneous
    equality, independence of functional extensionality, case studies
    on System F, dependent lambda calculus, pattern matching, linear
    types.
  \end{enumerate}

  \newpage

  \section{Syntax}
  The following shows a typical intrinsically-scoped syntax of System F:
  \ExampleCode\FUnsortedSyntax

  \texttt{Type}s are indexed by the number of free type variables \texttt{n}.
  \texttt{Expr}essions are additionally indexed by the number of free expression variables \texttt{m}.
  Variables are represented as DeBruijn-indices, where \texttt{Fin n} is the type of \texttt{n} elements.

  This style of syntax comes with multiple drawbacks:
  \begin{itemize}
  \item Bad scaling for automation
  \item Multiple substitions require interaction lemmas
  \end{itemize}

  % Instead we define a single term type which is indexed by the sort of
  % syntax, i.e.\ whether the term is a kind, type, or expression:
  To avoid these drawbacks, we define a single term type \texttt{S~⊢~s} that is indexed by its sort \texttt{s},
  i.e.\ whether the term is a kind, type, or expression, and the sorts of its free variables \texttt{S}:

  \LibCode\KSortTy

  \AppCode\FSort


  \LibCode\KVariables
  \AppCode\FSyntax
  The variable constructor \texttt{'\_} takes a sorted DeBruijn-index \texttt{S~∋~s}.
  Since the \texttt{Sort}s in \texttt{S} are restricted to the sort-type \texttt{Var}, it is not possible to construct kind-variables.

  \LibCode\KTerms
  \AppCode\FTerms

  \section{Substitution \& Renamings}
  Our sort-indexed syntax allows us to define substitutions and
  renamings which replace variables of all sorts simultaneously:

  \ExampleCode\FExampleSubRen

  A renaming \texttt{S₁ →ᵣ S₂} maps variables from \texttt{S₁} to
  variables from \texttt{S₂}.
  Similarly a substitution \texttt{S₁ →ₛ S₂} maps variables from \texttt{S₁} to
  terms from \texttt{S₂}.
  We consider these operations \emph{very parallel}, because they do not just replace all variables of one sort in parallel, but they replace all variables of all sorts in parallel.

  \ExampleCode\FExampleSub

  This representation has the benefit that there is no combinatory
  explosion of substitutions and renamings, e.g.\ no extra lemmas have to be
  proved between an expression-in-expression and a type-in-expression
  substitution, because both are simply substitutions.


  \subsection{Library}
  \LibCode\KKit
  \LibCode\KMap
  \LibCode\KAp
  \LibCode\KExt
  \LibCode\KLift
  \LibCode\KId
  \LibCode\KSingle
  \LibCode\KWeaken
  \LibCode\KEq
  \LibCode\KFunExt
  \LibCode\KIdLift
  % \LibCode\KIdLiftProof
  \LibCode\KKitNotation
  \LibCode\KTraversal
  \LibCode\KKitInstances
  \LibCode\KKitOpen
  \subsection{System F}
  \AppCode\FTraversalOp
  \AppCode\FTraversalId
  \AppCode\FTraversalIdProofInteresting
  % \LibCode\FTraversalIdProofRest
  \AppCode\FTraversal

  \section{Map Composition}
  \subsection{Library}
  \LibCode\KWkKit
  \LibCode\KWkKitInstances
  \LibCode\KComposeKit
  \LibCode\KComposition
  \LibCode\KComposeKitAp
  % \LibCode\KComposeKitApProof
  \LibCode\KDistLiftCompose
  % \LibCode\KDistLiftComposeProof
  \LibCode\KComposeKitNotation
  \LibCode\KComposeTraversal
  \LibCode\KCommLiftWeaken
  % \LibCode\KCommLiftWeakenProof
  \LibCode\KCommLiftWeakenTraverse
  % \LibCode\KCommLiftWeakenTraverseProof
  \LibCode\KComposeKitInstances
  \LibCode\KComposeKitInstancesConcrete
  \LibCode\KWeakenCancelsSingle
  % \LibCode\KWeakenCancelsSingleProof
  \LibCode\KWeakenCancelsSingleTraverse
  % \LibCode\KWeakenCancelsSingleTraverseProof
  \LibCode\KDistLiftSingle
  % \LibCode\KDistLiftSingleProof
  \LibCode\KDistLiftSingleTraverse
  % \LibCode\KDistLiftSingleTraverseProof
  \subsection{System F}
  \AppCode\FAssoc
  \AppCode\FAssocProofInteresting
  % \AppCode\FAssocProofRest
  \AppCode\FComposeTraversal

  \section{Types \& Contexts}
  \subsection{Library}
  \LibCode\KTypeSorts
  \LibCode\KTypes
  \LibCode\KContextHelper
  \LibCode\KContexts
  \LibCode\KContextLookup
  \subsection{System F}
  \AppCode\FTypes

  \section{Typing}
  \subsection{Library}
  \LibCode\KVariableTyping
  \LibCode\KTyping
  \LibCode\KTypingKit
  \LibCode\KMapTyping
  \LibCode\KLiftTyping
  % \LibCode\KLiftTypingProof
  \LibCode\KSingleTyping
  % \LibCode\KSingleTypingProof
  \LibCode\KTypingNotation
  \LibCode\KTypingTraversal
  \LibCode\KTypingInstances
  % \LibCode\KTypingTraversalNotation
  \subsection{System F}
  \AppCode\FTyping
  \AppCode\FTypingInst
  \AppCode\FPreserve
  \AppCode\FTypingTraversal

  \section{Semantics}
  % \AppCode\FValues
  \AppCode\FReduction

  \section{Subject Reduction}
  \AppCode\FSubjectReduction
  \AppCode\FSubjectReductionProofInteresting
  % \FSubjectReductionProofRest

  \section{Related Work}
  Autosubst \cite{DBLP:conf/cpp/StarkSK19, DBLP:conf/itp/SchaferTS15}.
  Kits \cite{DBLP:journals/jar/BentonHKM12, unpublished:mcbride2005kits}.
  Kits linear \cite{DBLP:journals/corr/abs-2005-02247}.
  Universe \cite{DBLP:journals/pacmpl/AllaisA0MM18}.

  % \bibliographystyle{ACM-Reference-Format}
  \bibliography{paper}

  \clearpage
  \appendix
\end{document}